\documentclass[conference]{IEEEtran}
\IEEEoverridecommandlockouts
% The preceding line is only needed to identify funding in the first footnote. If that is unneeded, please comment it out.
\usepackage{cite}
\usepackage{amsmath,amssymb,amsfonts}
\usepackage{algorithmic}
\usepackage{graphicx}
\usepackage{textcomp}
\usepackage{xcolor}
\def\BibTeX{{\rm B\kern-.05em{\sc i\kern-.025em b}\kern-.08em
    T\kern-.1667em\lower.7ex\hbox{E}\kern-.125emX}}
\begin{document}

\title{Parallel Domain Decomposition Techniques Applied to Multi-Variate Functional Approximation of Discrete Data\\
%{\footnotesize \textsuperscript{*}Note: Sub-titles are not captured in Xplore and should not be used}
%\thanks{Identify applicable funding agency here. If none, delete this.}
\thanks{Early Career Research Program, Department of Energy, US}
}


\author{\IEEEauthorblockN{1\textsuperscript{st} Vijay S. Mahadevan}
\IEEEauthorblockA{\textit{Mathematics and Computational Science Division} \\
\textit{Argonne National Laboratory}\\
Lemont, IL, 60439, USA \\
mahadevan@anl.gov}
\and
\IEEEauthorblockN{2\textsuperscript{nd} Thomas Peterka}
\IEEEauthorblockA{\textit{Mathematics and Computational Science Division} \\
\textit{Argonne National Laboratory}\\
Lemont, IL, 60439, USA \\
tpeterka@mcs.anl.gov}
\and
\IEEEauthorblockN{3\textsuperscript{rd} Iulian Grindeanu}
\IEEEauthorblockA{\textit{Mathematics and Computational Science Division} \\
\textit{Argonne National Laboratory}\\
Lemont, IL, 60439, USA \\
iulian@anl.gov}
\and
\IEEEauthorblockN{4\textsuperscript{th} Youssef Nashed}
\IEEEauthorblockA{\textit{Stats Perform}\\
Chicago, IL, 60601, USA \\
youssef.nashed@statsperform.com}
}

\maketitle

\begin{abstract}
Compactly expressing large-scale datasets through multivariate functional approximations (MFA) can be critically important for analysis and visualization to drive scientific discovery. This paper presents a data and domain partitioning approach to scalably compute a MFA representation, by reducing the total work per task in combination with a nonlinear Schwartz-type, inner-outer iterative scheme for converging the interface data. For the underlying MFA, we utilize a tensorial expansion of non-uniform B-spline (NURBS) basis to adaptively reduce the functional approximation error in the input data. While previous work on adaptive NURBS-based MFA has been proven successful, the computational complexity for encoding large datasets on a single process can be prohibitive. We demonstrate effectiveness of the presented approach with an overlapping Jacobi-Schwartz based domain decomposition solver, with a nonlinear accelerator such as L-BFGS or Krylov (nCG, L-GMRes) to minimize the subdomain error residuals obtained from decoding the MFA, and more specifically to resolve the discontinuities at boundaries. The analysis of the presented scheme for some analytical and real scientific datasets in 1-D and 2-D are also presented. Additionally, scalability studies are also shown for some real-world 2-d datasets to evaluate the parallel speedup of the algorithm on large clusters.
\end{abstract}

\begin{IEEEkeywords}
functional approximation, domain decomposition, scalable methods
\end{IEEEkeywords}

\section{Introduction}

Large scale discrete data analysis from various scientific computational simulations often require high-order continuous functional representations that have to be evaluated anywhere in the domain. Such expansions described as Multivariate Functional Approximations (MFA) in arbitrary dimensions \cite{nurbs-book} allow the original discrete data to be compressed, and expressed in a compact closed form in addition to supporting higher-order derivative queries. One particular option is to use NURBS bases for the MFA encoding of scattered data \cite{peterka-mfa}. Due to the potentially large datasets that need to be encoded into a MFA, the need for computationally efficient algorithms (in both time and memory) to partition the work into subtasks is critically important. In the current paper, we utilize domain decomposition techniques \cite{smith-ddm} with data partitioning strategies to produce scalable algorithms to adaptively compute the MFA to reproduce a given dataset within user-specified tolerances. In such partitions, it is imperative to ensure that the continuity of the data across subdomains is maintained and consistent with the degree of the underlying bases used in the MFA. The interface data at subdomain boundaries are iteratively converged through an outer Schwartz-type iterative scheme in order to ensure continuity and overall error that stays bounded as number of subdomains are increased (subdomain size decreases).

%{\color{red}THIS IS A DRAFT}

%\begin{itemize}
%	\item Talk about MFA and how it can be used to approximation discrete solution data. Reference previous work.
%	\item Provide motivations on why this is necessary especially for large datasets
%	\item Literature survey of other work for parallel interpolation and compression of data
%	\item What are the other approaches to address this issue; pros and cons
%\end{itemize}

Domain decomposition techniques for parallel interpolation of scattered data has been explored previously with Radial Basis Functions (RBF) \cite{mai-approx-rbf}, yielding good scalability to create a MFA that closely replicates underlying profile. Additionally, use of restricted Additive-Schwartz preconditioners with GMRes iterative solvers have been shown to be scalable \cite{yokota-rasm-rbf} for such problems. However, using NURBS-basis to compute MFA in parallel, while maintaining higher-order continuity across subdomains has not been explored previously in this context. To overcome the issues with discontinuities along NURBS patches, \cite{zhang-nurbs-continuity} have proposed to use a gradient projection scheme to constrain the value (G0), the gradient (G1) and the hessian (G2) at a small number of test points for optimal shape recovery. 



While it is also possible to create such a constrained recovery during the actual post-processing stage i.e., decoding of the MFA through blending techniques \cite{grindeanu-blending}, the underlying MFA representation would remain discontinuous, and would become more so with increasing number of subdomains. In contrast, we propose an extension to the constrained solvers used by \cite{zhang-nurbs-continuity, xu-jahn-discrete-adjoint} by utilizing a recursive domain-decomposition (DD) based outer-inner iterative scheme to resolve continuity prescriptions as required by the user. The outer iteration utilizes a flavor of the restricted Additive-Schwartz scheme, Jacobi-Schwartz \cite{smith-ddm}, with an efficient, inner subdomain solver using L-BFGS as used in \cite{zheng-bo-bspline-bfgs}, or Krylov-type schemes to minimize the decoded residual within acceptable error tolerances. This DD solver has low memory requirements that scales with growing subdomains, and only imposes nearest neighbor communication of the interface data once per outer iteration. 


\section{Approach}

\begin{itemize}
	\item What are we proposing and why this can be a stable technique to recover high-order continuity in parallel ?
	\item Give context about DD methods and how ASM in this context makes sense 
	\item Refer to \cite{smith-ddm} and \cite{ddm-rbf-fast} as well and write out the equations with \cite{nurbs-book} help
\end{itemize}

\subsection{Methodology}

Explain the iterative scheme in terms of the underlying equations and how the boundary terms are resolved through a global ASM method. First start with 1-d and talk about extensions in the scheme to allow arbitrary dimensional solver framework.

\begin{eqnarray} \label{asm}
L_1U^{n}_1=f_1(x) \in \Omega_1  \\
L_1U^{n}_1=g(x) \in \partial\Omega_1 \\ 
U^{n}_1 = U^{n-1}_2 \in \partial\Omega_1 \\
L_2U^{n}_2=f_2(x) \in \Omega_2 \\
L_2U^{n}_2=g(x) \in \partial\Omega_2 \\
U^{n}_2 = U^{n-1}_1 \in \partial\Omega_2
\end{eqnarray}


\subsection{Implementation}

Talk about DIY and Python implementations for the code. Provide algorithmic references here as well.

\section{Results}

Explain about the common problem datasets that we are going to be using and why they have been chosen.

Please note sections \ref{AA}--\ref{SCM} below for more information on 
proofreading, spelling and grammar.

\subsection{1-d Results}\label{AA}

Problem setups
\begin{itemize}
  \item Sine
  \item Sinc
  \item S3D
  \item Nuclear data
\end{itemize}

\begin{itemize}
	\item Use the first two problems to measure convergence in parallel as number of domains increase
	\item Talk about adaptivity and resolution of data even for highly varying problem data.
\end{itemize}

\subsection{2-d Results}

Problem setups
\begin{itemize}
	\item Sinc
	\item Nek5000
	\item S3D
\end{itemize}


\begin{itemize}
	\item Adaptive algorithms to resolve data and provide compression
	\item DD scheme with ASM in combination with adaptivity.
	\item Discuss about complications and potential ways to enforce continuity. (a) Use decoded data, (b) Use control point space across interface
\end{itemize}

\subsection{Parallel Scalability}

Showcase some scalability results on Bebop for the 2-d problem; Take S3D and CESM datasets, along with artificial sinc combination functions.

How does the nearest neighbor communication stay bounded ?

%\paragraph{Positioning Figures and Tables} Place figures and tables at the top and 
%``Fig.~\ref{fig}'', even at the beginning of a sentence.

%\begin{table}[htbp]
%\caption{Table Type Styles}
%\begin{center}
%\begin{tabular}{|c|c|c|c|}
%\hline
%\textbf{Table}&\multicolumn{3}{|c|}{\textbf{Table Column Head}} \\
%\cline{2-4} 
%\textbf{Head} & \textbf{\textit{Table column subhead}}& \textbf{\textit{Subhead}}& \textbf{\textit{Subhead}} \\
%\hline
%copy& More table copy$^{\mathrm{a}}$& &  \\
%\hline
%\multicolumn{4}{l}{$^{\mathrm{a}}$Sample of a Table footnote.}
%\end{tabular}
%\label{tab1}
%\end{center}
%\end{table}

%\begin{figure}[htbp]
%\centerline{\includegraphics{fig1.png}}
%\caption{Example of a figure caption.}
%\label{fig}
%\end{figure}

\section{Conclusion}

\begin{itemize}
	\item What did we implement to enhance speedup of the MFA framework and did we preserve accuracy of the underlying method ?
	\item Did we speedup the actual computation by performing DD with ASM global iterations for some of the problem data ?
	\item Does the method scale as a function of domains and problem size ? 
	\item What advantages does it provide for fix-up schemes that can be used in a post-processing step (ref Iulian's blending idea) ?
	\item Future extensions to T-splines and local adaptivity and potential complications involved
\end{itemize}

\section*{Acknowledgment}

Acknowledge funding sources: DOE ECRP and others ?


\begin{thebibliography}{00}

\bibitem{nurbs-book} Piegl, Les, and Wayne Tiller. The NURBS book. Springer Science \& Business Media, 2012.

\bibitem{peterka-mfa} Peterka, Tom, S. G. Youssef, Iulian Grindeanu, Vijay S. Mahadevan, Raine Yeh, and Xavier Tricoche. "Foundations of multivariate functional approximation for scientific data." In 2018 IEEE 8th Symposium on Large Data Analysis and Visualization (LDAV), pp. 61-71. 2018.

\bibitem{nashed-rational} Nashed, Youssef SG, Tom Peterka, Vijay Mahadevan, and Iulian Grindeanu. "Rational Approximation of Scientific Data." In International Conference on Computational Science, pp. 18-31. Springer, Cham, 2019.

\bibitem{grindeanu-blending} Grindeanu, Iulian, Tom Peterka, Vijay S. Mahadevan, and Youssef SG Nashed. "Scalable, High-Order Continuity Across Block Boundaries of Functional Approximations Computed in Parallel." In 2019 IEEE International Conference on Cluster Computing (CLUSTER), pp. 1-9. IEEE, 2019.

%\bibitem{raine-knot-placement} Yeh, R., Nashed, Y., Peterka, T., Tricoche, X.: Fast Automatic Knot Placement Method for Accurate B-spline Curve Fitting. Submitted to Journal of Computer-Aided Design, 2020.

\bibitem{zheng-bo-bspline-bfgs} Zheng, Wenni, Pengbo Bo, Yang Liu, and Wenping Wang. "Fast B-spline curve fitting by L-BFGS." Computer Aided Geometric Design 29, no. 7 (2012): 448-462.

\bibitem{mai-approx-rbf} Mai-Duy, Nam, and Thanh Tran-Cong. "Approximation of function and its derivatives using radial basis function networks." Applied Mathematical Modelling 27, no. 3 (2003): 197-220.

\bibitem{yokota-rasm-rbf} Yokota, Rio, Lorena A. Barba, and Matthew G. Knepley. "PetRBF—A parallel O (N) algorithm for radial basis function interpolation with Gaussians." Computer Methods in Applied Mechanics and Engineering 199, no. 25-28 (2010): 1793-1804.

\bibitem{smith-ddm} Smith, Barry, Petter Bjorstad, and William Gropp. Domain decomposition: parallel multilevel methods for elliptic partial differential equations. Cambridge university press, 2004.

\bibitem{ddm-rbf} Li, Jichun, and Y. C. Hon. "Domain decomposition for radial basis meshless methods." Numerical Methods for Partial Differential Equations: An International Journal 20, no. 3 (2004): 450-462.

\bibitem{ddm-rbf-fast} Beatson, Richard K., W. A. Light, and S. Billings. "Fast solution of the radial basis function interpolation equations: Domain decomposition methods." SIAM Journal on Scientific Computing 22, no. 5 (2001): 1717-1740.

\bibitem{xu-jahn-discrete-adjoint} Xu, Shenren, Wolfram Jahn, and Jens‐Dominik Müller. "CAD‐based shape optimisation with CFD using a discrete adjoint." International Journal for Numerical Methods in Fluids 74, no. 3 (2014): 153-168.

\bibitem{zhang-nurbs-continuity} Zhang, Xingchen, Yang Wang, Mateusz Gugala, and Jens-Dominik Müller. "Geometric continuity constraints for adjacent NURBS patches in shape optimisation." In ECCOMAS Congress, vol. 2, p. 9316. 2016.

\end{thebibliography}

%\vspace{12pt}
%\color{red}
%IEEE conference templates contain guidance text for composing and formatting conference papers. Please ensure that all template text is removed from your conference paper prior to submission to the conference. Failure to remove the template text from your paper may result in your paper not being published.

\end{document}
